\documentclass[a4paper,11pt]{article}


\title{MA3H0 Assignment 3}
\author{Joshua Rydell}
\usepackage{graphicx}
\usepackage{framed}
\usepackage{amsmath}
\usepackage{amssymb}
\usepackage{amsfonts}
\usepackage{cite}
\usepackage{mathrsfs}
\usepackage{array}
\usepackage[numbers]{natbib}
\bibliographystyle{plainnat}
\usepackage{cite}
\usepackage{xcolor}


\usepackage{amsthm} %  package used to make the theorem environments work.
% code below sets up new theorem environments
\theoremstyle{plain} % this sets the style for all new environments created using \newtheorem to have the "theorem" style, which as a bold title, italic text and vertical space above and below it. 
\newtheorem{thm}{Theorem}[section] 
\newtheorem{lem}[thm]{Lemma} 
\newtheorem{prop}[thm]{Proposition} 
\newtheorem*{cor}{Corollary} 
\newtheorem*{claim}{Claim} 

\theoremstyle{definition} % this sets the style for all new environments created using \newtheorem to have the "definition" style, which as a bold title, upright text and vertical space above and below it.
\newtheorem{defn}{Definition}[section]
\newtheorem{eg}{Example}[section]


\theoremstyle{remark} % this sets the style for all new environments created using \newtheorem to have the "remark" style, which as an italic non-bold title, upright text and no extra vertical space above and below it.
\newtheorem*{rem}{Remark} 
\newtheorem{case}{Case}

% note that there is already an in-built "proof" environment which we do not need to create


\newcommand{\N}{\mathbb{N}}
\newcommand{\Lc}{\mathcal{L}}
\newcommand{\brac}[1]{\left ( #1  \right )}
\newcommand{\abs}[1]{\left | #1  \right | }
\newcommand{\qu}{\underline q}
\newcommand{\qo}{\bar{q}}
\newcommand{\du}{\underline d}
\newcommand{\ip}[2]{\left \langle #1 , #2 \right \rangle}
\newcommand{\dob}{\bar{d}}
\newcommand{\restr}[2]{\left.#1\right|_{#2}}

\begin{document}  
\maketitle
We consider the problem:
\begin{equation} \label{problem}
\begin{cases}
-\partial_x(d(x) \partial_x \tilde u(x)) + q(x) \tilde u(x) = f(x) & \text{ in } \Omega = (0,1), \\
\partial_x \tilde u(0) = \partial_x \tilde u(1) = 0,
\end{cases}
\end{equation}
with $f \in L^2
(\Omega), d, q \in C^0 (\Omega)$. Assume that there exist positive constants $ \dob, \du, \qu, \qo$
such that
\[ 0 < \du \leq d(x) \leq \dob, \; \; \; 0 < \qu \leq q(x) \leq \qo  \; \; \; \forall x \in \overline \Omega. \]
\begin{enumerate}
\item To convert (\ref{problem}) to a variational problem we use integration by parts, let $\varphi \in H^1(\Omega)$. We then calculate:
\[\int_0^1 f(x) \varphi(x) dx = \int_0^1 -\partial_x(d(x) \partial_x \tilde u(x)) \varphi(x) + q(x) \tilde u(x) \varphi(x) dx  \] 
\[=\int^1_0d(x)   \partial_x \tilde u(x)\partial_x \varphi(x)  + 
q(x) \tilde u(x) \varphi(x)dx - \left [d(x)\varphi(x)  \partial_x \tilde u(x) \right]^1_0\]
Note that $ \left [\varphi(x) d(x) \partial_x \tilde u(x) \right]^1_0 = 0$ due to zero Neumann boundary conditions. Therefore we have shown that (\ref{problem}) can be formulated as:
\[ \int^1_0  d(x)  \partial_x \tilde u(x)\partial_x \varphi(x) + q(x) \tilde u(x) \varphi(x)dx =  \int_0^1 f(x) \varphi(x) dx,\]
for all  $\varphi \in H^1(\Omega)$.

\item We define the bilinear form $a : H^1(\Omega) \times H^1(\Omega) \to \mathbb R$ and a linear form $\ell : H^1(\Omega) \to \mathbb R$ via :
\[a(u,v) = \int^1_0 d(x)\partial_x v(x)   \partial_x u(x) + q(x)  u(x) v(x)dx , \]
\[\ell(v) = \int_0^1 f(x) v(x) dx .\]

We first show that $a$ is bounded. We calculate via the triangle inequality:
\[ |a(u,v)| = \abs{\int^1_0 d(x)\partial_x v(x)   \partial_x u(x) + q(x)  u(x) v(x)dx} \]
\[\leq \max \{\dob, \qo\} \abs{\int^1_0 \partial_x v(x)   \partial_x u(x) + u(x) v(x)dx} \] 
\[=\max \{\dob, \qo\} \abs{\ip{u}{v}_{H^1(\Omega)}} \leq \max \{\dob, \qo\} \|u\|_{H^1(\Omega)}\|v\|_{H^1(\Omega)}. \]
where to get the final line we applied Cauchy Schwartz.
Therefore $a$ is bounded.

Similarly, to show $a$ is coercive :
\[a(v,v) = \int^1_0 d(x)(\partial_x v(x))^2+ q(x)(v(x))^2dx \] 
\[\geq  \min \{\qu, \du \} \int^1_0(\partial_x v(x))^2+(v(x))^2dx  = \min \{\qu, \du \}\|v\|_{H^1(\Omega)}^2.\] 
Thus $a$ is also coercive.

Finally we show that $\ell$ is bounded in the $H^1$ norm:
\[\abs{\ell(v)} = \abs{\int_0^1 f(x) v(x) dx} = \abs{\ip{f}{v}_{L^2(\Omega)}} \] 
\[\leq  \|f\|_{L^2(\Omega)}\|v\|_{L^2(\Omega)} \leq \|f\|_{L^2(\Omega)}\|v\|_{H^1(\Omega)},\]
where again we used Cauchy-Schwartz in the last line.
\item We recall the definition of the test functions $\psi_j, \; j \in \{0,\dotsc, N_h\}$: We define $h = 1/N_h$, set $x_j = jh$. Then 
\[\psi_j(x) = 
\begin{cases} 
(x-x_{j-1})/h & x\in [x_{j-1}, x_j], \\
(x_{j+1}-x)/h & x \in [x_j, x_{j+1}], \\
0 & \text{ otherwise.} 
\end{cases} \] 

We then calculate for $j \in \{1,\dotsc, N_h-1 \}$ that
\[a(\psi_j,\psi_j) = \int_0^1d(x)(\partial \psi_j(x))^2 + q(x)(\psi_j(x))^2dx \] 
\[= \frac{1}{h^2} \int_{x_{j-1}}^{x_j} d(x) + q(x)(x-x_{j-1})^2dx + \frac{1}{h^2}\int_{x_{j}}^{x_{j+1}} d(x) + q(x)(x_{j+1}-x)^2dx.\]
Furthermore, we calculate
\[a(\psi_0,\psi_0) =  \frac{1}{h^2}\int_{0}^{h} d(x) + q(x)(h-x)^2dx, \]
\[a(\psi_{N_h},\psi_{N_h}) =\frac{1}{h^2} \int_{1-h}^1 d(x) + q(x)(x-1+h)^2dx . \]

We again calculate that for  $j \in \{1,\dotsc, N_h \}$
\[a(\psi_{j-1}, \psi_j) = a(\psi_j,\psi_{j-1}) = \frac{1}{h^2}\int_{x_{j-1}}^{x_j} -d(x) + q(x)(x_j-x)(x-x_{j-1})dx. \]

We also have that for $ i \neq j-1, j , j+1$:
\[ a(\psi_i,\psi_j) = 0. \]

We also note:
\[\ell(\psi_j) = \frac 1 h \int_{x_{j-1}}^{x_j} f(x)(x-x_{j-1})dx +\frac 1 h \int_{x_{j}}^{x_{j+1}} f(x)(x_{j+1}-x)dx. \]
We set 
\[A =\begin{pmatrix}
a(\psi_0,\psi_0) & \hdots  & a(\psi_0,\psi_{N_h}) \\
\vdots & \ddots & \vdots \\
a(\psi_{N_h},\psi_0) & \hdots  & a(\psi_{N_h},\psi_{N_h}) 
\end{pmatrix}, \; 
B = \begin{pmatrix}
\ell(\psi_0) \\
\vdots \\
\ell(\psi_{N_h})\\
\end{pmatrix}
\] 
\[U = 
\begin{pmatrix}
u_{0}\\
\vdots \\
u_{N_h} \\
\end{pmatrix} \]
Then the discrete problem can be solved by solving the tridiagonal matrix system :
\[AU = B \]
\item By applying Cea (lemma 4.13) in the notes, we calculate 
\[ \| \tilde u - u_h \|_{H^1(\Omega)} \leq \frac{\max \{\dob, \qo\} }{\min \{\qu, \du \}} \| \tilde u - P_h(\tilde u)  \|_{H^1(\Omega)}, \]
where $P_h(\tilde u) = \sum^{N_h}_{j=0} \tilde u (x_j)\psi_j \in V_h$.
We then apply the interpolation estimate (lemma 4.17) to get that there is some $\tilde C$ such that:
\[\frac{\max \{\dob, \qo\} }{\min \{\qu, \du \}} \| \tilde u - P_h(\tilde u)  \|_{H^1(\Omega)}  \leq \tilde C h \frac{\max \{\dob, \qo\} }{\min \{\qu, \du \}} \| \tilde u  \|_{H^2(\Omega)}.  \]
The result follows from taking $C = \tilde C  \frac{\max \{\dob, \qo\} }{\min \{\qu, \du \}}$.
\item By plugging in the integrals from the solution to question three into Wolfram Alpha, we calculate that for the given $q(x),d(x),f(x)$:
\[a(\psi_0,\psi_0) =  \frac{7h}{20},\]
\[a(\psi_{N_h},\psi_{N_h}) = \frac{h}{20}(N_h^2-N_h+7),\]
\[a(\psi_j,\psi_j) = \frac{hj^2+7h}{10}, \; \; \; j \in \{1, \dotsc, N_h-1\},  \]
\[a(\psi_j,\psi_{j-1}) = a(\psi_{j-1},\psi_{j}) =  \frac{h}{20}(-j^2+j+3).\]

For the vector $B$, we calculate that :
\[\ell(\psi_0) =  \frac{24h^4-35h^3}{600}, \]
\[\ell(\psi_{N_h}) =  \frac{h^4}{25}(10N_h^3 - 10 N_h^2+5N_h -1) - \frac{7h^3}{120}(6N_h^2-4N_h+1)),\]
\[\ell(\psi_j) =  \frac{2h^4}{5}(2j^3+j)-\frac{7h^3}{60}(6j^2+1)  \; \; \; j \in \{1, \dotsc, N_h-1\}.\]

In the jupyter notebook we calculate the EOC for the $\| \cdot \|_\infty$ norm and see that it is $\approx 2$.
\end{enumerate}
\end{document}
